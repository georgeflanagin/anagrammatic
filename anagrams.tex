\documentclass[letterpaper, 11pt]{article}
\usepackage{cmap}
\usepackage{urtechdoc}
\usepackage{urfooter}
\providecommand{\anagrammar}{A\textsc{nagrammar}\xspace}


\SetWatermarkText{Review Copy}
\SetWatermarkFontSize{5cm}
\SetWatermarkAngle{58}
\SetWatermarkColor[gray]{0.97}

\title{An Algorithm to Find Anagrams Using Python}
\author{George Flanagin\\University of Richmond\\\lit{gflanagin@richmond.edu}}
\date{\today}

\setlength{\parskip}{0.8em}
\setlength{\parindent}{0.8em}

\newcommand{\D}{$\mathfrak{D}$\xspace}
\newcommand{\Dp}{$\mathfrak{D}'$\xspace}
\newcommand{\Dpp}{$\mathfrak{D}''$\xspace}
\newcommand{\w}{$w$\xspace}
\newcommand{\aw}{$\overrightarrow{w}$\xspace}
\newcommand{\awp}{$\overrightarrow{w'}$\xspace}
\newcommand{\ra}{$\overrightarrow{r}$\xspace}
\newcommand{\rap}{$\overrightarrow{r'}$\xspace}

\sloppy

\begin{document}
\maketitle
\begin{abstract}
Anagrams are fun and difficult. They are an excellent way to explore
Python's data structures, and an excuse to develop a few new ones
expressly for this purpose. This paper defines an efficient algorithm
for finding all anagrams of a given collection of letters using the
Python Standard Library for the base code, and the system dictionary,
or some other list of words,
as a source for allowed words. 

This program is the \anagrammar. Its source code and this documentation
may be found in \href{https://github.com/georgeflanagin/anagrammatic}
{https://github.com/georgeflanagin/anagrammatic}

\end{abstract}

\tableofcontents
\listoffigures

\newpage
\pagewiselinenumbers
\section{Getting started}
\quickquote{Of course just about every pronounceable combination
of five letters has been used to spell or misspell something somewhere, at
some point in history.}{Donald E. Knuth, 2011\\\emph{Combinatorial
Algorithms, Part 1}, p. 519\\answer to Problem 25 in Section 7}

\subsection{Definitions}
My interest in anagrams started later in life, at 37 years old. An episode of
\emph{The Simpsons} named ``Lisa's Rival'' aired 11 September 1994,
and in it Lisa visits a new girl in the school, Allison
Taylor. At Allison's house, Lisa is asked to join the Taylors in a
specialized game of anagrams in which the names of famous people
are rearranged to form descriptive anagrams: \emph{Alec Guinness}
becomes \emph{genuine class} in the show's example. Lisa was overwhelmed.

To avoid offending or upsetting readers, I will be using my own
name as the basis for the examples in this paper. Mercifully for
the contruction of examples, my name consists only of common letters
in English, and it has a useful proportion of vowels. Let's call
the starting point the \emph{target phrase}, or just the \emph{target}.

To begin, we must have a definition of an anagram. Anagrams
always appear in pairs.  It makes no sense to say ``\emph{George
Flanagin} is an anagram," without stating its partner phrase, and the
one most suited to Lisa Simpson's game may be
\emph{long fearing age}. 

The technical properties of an anagram are:

\begin{enumerate}
\item Two phrases are anagrams of each other iff all the letters of
one phrase are present in the other, the same counts for each letter
present.
In the case of \emph{George Flanagin}, its
anagrams must have exactly three \emph{g}-s, two each of \emph{a},
\emph{e}, and \emph{n}, and one each of \emph{f}, \emph{i}, \emph{l},
\emph{o}, and \emph{r}.  Fourteen letters.

\item Each phrase in a pair of anagrams must consist of one or more complete words in some dictionary.
Exactly which dictionary is a point of {\ae}sthetics, as is whether
one should consider single letter words such as \lit{a} in English
\emph{and} Spanish, \lit{I} in English, and \lit{y}, \lit{o}, \lit{e}, \lit{u} in Spanish.
For the purpose of this paper, we will ignore diacritics.

\item Additional rules are sometimes applied. Common constraints are:

\begin{itemize}
\item A maximum or minimum number of words in the anagrams of the original phrase,
and the related metric, minimum word length.

\item The elimination or inclusion of proper nouns in the derived
anagrams.

\item A lack of words shared with the original phrase. More discussion of this 
constraint lies ahead.
\end{itemize}
\end{enumerate}

Fortunately, none of the above boundaries is difficult to selectively
enforce in a program. 

\subsection{The big picture of the search}
\label{sec:bigpicture}

At first, the search seems terribly complicated, and it is certainly
not trivial. Everyone who has dealt with combinatorics knows how
quickly the the number of subsets grows, $16383$ in the case of
fourteen letters (\ie $2^{14}-1$), and the number of partitions, the fourteenth Bell
number, is $27,644,437$.\footnote{The Bell numbers are OEIS sequence
A000110, \lit{https://oeis.org/A000110}}

The constraint is the dictionary. The largest one generally used
in computing is the list of words that ships as the spelling
dictionary with Linux, and it has a mere $479826$ words. In any
search for anagrams except \emph{the quick brown fox jumped over
the lazy dog}, we will eliminate all the words that contain any
letter not found in the target. In the case of our example, we can
eliminate not only all the words that begin with $[b-d,h,j,k,m,n,p,q,s-z]$,
but all the words that contain one of those letters. Doing no
programming at all, we can put the upper limit on the number of
words to examine as $2777$ if only lower case words are considered,
and $5032$ if we allow words that normally have capitals in
English.\footnote{These numbers are the results of the two na\"ive
regular expression searches of the Linux spelling dictionary with
\lit{\textasciicircum[aegnfilor]\textbackslash+\$} in case sensitive
and case insensitive modes.} These are the upper bounds, because
the figures are not derived by looking for the words that have no
more of each letter than are found in the target.

For many programmers, the following is a useful abstraction to
consider how we go about constructing both \emph{an} anagram, and
\emph{all} anagrams.  Consider the target phrase and bit-string of
the same length. In Figure~\ref{fig:bitmask}, we see two anagrams,
\emph{george}+\emph{flanagin} and \emph{long}+\emph{fearing}+\emph{age},
where each character is ($1$) or is not ($0$) used. For both of the
anagrams, there is exactly one $1$ in each column, and this tabulation
is similar to the way that you might search for anagrams with a
pile of Scrabble\CircleR\xspace letters. The equivalent big endian
numeric represenations are shown in the right column.

\begin{figure}
\begin{center}
\begin{tabular}{lccccccccccccccr}
\toprule
 &g&e&o&r&g&e&f&l&a&n&a&g&i&n&\\
\midrule
george&1&1&1&1&1&1&0&0&0&0&0&0&0&0&16128\\
flanagin&0&0&0&0&0&0&1&1&1&1&1&1&1&1&255\\
\midrule
long&1&0&1&0&0&0&0&1&0&1&0&0&0&0&10320\\
fearing&0&1&0&1&1&0&1&0&1&0&0&0&1&1&5795\\
age&0&0&0&0&0&1&0&0&0&0&1&1&0&0&268\\
\bottomrule
\end{tabular}
\caption{The relationship between anagrams and bit masks}
\label{fig:bitmask}
\end{center}
\end{figure}

\begin{figure}
\begin{tabular}{lp{0.8\textwidth}}
\toprule
Symbol&Use and meaning\\
\midrule
$w$, $w_n$&word or words from a dictionary.\\

$\mathfrak{S}$, $\mathfrak{T}$&phrases for which are finding anagrams.\\

$\overrightarrow{w}$, $\overrightarrow{\mathfrak{S}}$&representations
of $w$ and $\mathfrak{S}$ where the letters have been sorted. In
the programming section, we will be using a standard lexical sort,
but this is unimportant as long as the same sort-order is used
throughout. If $w$ is \lit{loaf}, then $\overrightarrow{w}$ is
\lit{aflo}.\\

$w \leq \mathfrak{S}$&This expression is true iff $w$ can be
constructed from the letters in $\mathfrak{S}$. For example \lit{foal}
$\leq$ \lit{georgeflanagin}, and \lit{foal} $\leq$ \lit{loaf}. Note
that this expression is true or false without regard to whether the
letters in each term have been sorted.\\


$w_1 \oplus w_2$&The result is a collection of all the letters in
the two words, without preserving the order of the letters in each
word. The $\oplus$ operator was chosen over $+$ because the bare
plus sign is used as a string concatentation operator in many
programming languages, including Python.\\

$w_1 \ominus w_2$&As with the $oplus$ operation above, except that
we are removing all the letters in $w_2$ from $w_1$. This operation
is only defined (or meaningful) iff $w_2 \leq w_1$, otherwise in
the grammar of anagrams (anagrammar?), the statement is like dividing
by zero.\\

$w_1 \odot w_2$&This expression is true iff $w_1$ and $w_2$ are
anagrams, so expanding on the above examples, \lit{foal} $\odot$
\lit{loaf} is true, as is:

$($\lit{long} $\oplus$ \lit{fearing} $\oplus$ \lit{age}$)$ $\odot$
\lit{georgeflanagin}.\\

$r$, $r_n$&$r$ is for remainder, so when we perform a $\ominus$
operation, the result will be a value expressed as an $r$, so $r_1
= \mathfrak{S} \ominus w_1$\\

\D, \Dp, \emph{etc}.&Throughout, we will use \D to represent the
core dictionary, and \Dp and \Dpp to represent derived dictionaries
such that \Dpp $\subset$ \Dp $\subset$ \D, in other words, a filter.\\

\bottomrule
\end{tabular}
\caption{Symbols used to discuss anagrams}
\label{fig:anagramsymbols}
\end{figure}

\begin{samepage}
\section{Notation}

Anagrams suffer from being relatively easy to define
in English, yet having no obvious representation in symbols. Given
that programming languages are a type of symbolic representation,
we will benefit from having a strong symbolic notation for anagrams,
their parts, and their construction.
\end{samepage}

In Figure~\ref{fig:anagramsymbols} we can see the basic notation
that is invented for the purpose of this discussion. The other
symbols used, such as assignment, absolute value, and non-anagram
set operations, are expressed in conventional notation, and will
mean what you expect them to.

The notation suggests that our Python representation will need all
of these operators/operations:

\begin{enumerate}
\item Sorting the letters of a string to make an new string of the 
same length.

\item The always useful ``partial ordering'' operator, $\leq$, so that
we can determine if words can be used to make an anagram of the target
phrase.

\item A method to combine strings beyond concatenation, and a method
to remove letters from a string.

\item A method to filter the useful words in a dictionary to make
a new, subset dictionary.

\item A method of bookkeeping to allow us to track the anagrams that
have been found, and avoid searching the same path twice.
\end{enumerate}

In fact, these operations lead us directly to a discussion of the 
algorithm to find all anagrams. 

\newpage
\section{Algorithm}

\quickquote{Homer Jay, how do you keep your hair so rich and full?\\
Lather, rinse, and repeat. Always repeat.}{Homer Simpson\\D'oh-in'
in the Wind\\15 November 1998}

Given the small number of qualifying words, and the vast available
memory combined with the computational abilities of even bottom-self
computers, finding all anagrams of a phrase could be done by brute
force. That approach is not very satisfying. Instead, we are searching
for elegance and comprehension, two nouns that often appear
side-by-side.

\subsection{Pruning the dictionary}

Our first step is the elimination of all the words that are made
from incompatible collections of letters. In our grammar, we seek
to construct

\begin{equation}
\label{eq:deriveddictionaries}
\mathfrak{D}' := \{w:w \leq \mathfrak{S} \wedge w \in \mathfrak{D}\}
\end{equation}

This is often a small collection of words, and we know that 

\begin{equation}
\label{eq:wordlength}
\forall w: |w| \leq |\mathfrak{S}|
\end{equation}


\subsection{Trying the words}

It makes sense to start with the longest words in \Dp, a fact that
will guide us when we start the programming in the next section.
Each word in the dictionary will be subtracted from the target
phrase, and will leave a complementary set of remainders.

\begin{equation}
\label{eq:remainders}
\mathfrak{R} = \{\forall w : \mathfrak{S} \ominus w \}
\end{equation}

Equation~\ref{eq:remainders} is well suited to the list comprehension
construct in Python.

At this point, we should take note of the one-to-many relationship
between \aw and the dictionary words \w. Words that are anagrams
of each other share the same sorted representation of their letters,
so in a key-value look up table, if the keys are of the form \aw,
then they must support a list (tuple) of one or more \w-s as the
values.

While programming this data structure takes us into the shallow end
of the pool of rolling our own data structures later on, it does
mean that we do not need to try any of the words in the tuple to
make an anagram, we need only concern ourselves with the sorted
key. Consider this specific case: $acer \longrightarrow \{acre,
race, care\}$

The sorted representation, $acer$, is meaningless. But we can freely
substitute any of the three English words in an anagram that contains
one of them. Whether the collection of real words associated with
a key has one or more than one element, we only need to bother with the key.
Thus, Equation~\ref{eq:remainders} becomes the more manageable
expression seen here:

\begin{equation}
\mathfrak{R} = \{\forall \overrightarrow{w} : \mathfrak{S} \ominus \overrightarrow{w} \}
\end{equation}

We cannot neglect the fact that $\forall w: w \odot w$, or in plain English,
every word is an anagram of itself, so we must check to see whether each
remainder is a key in the dictionary and a complement of some other
key in the same dictionary. So before any recursive decent begins, we 
must check for the ``two word'' solution.\footnote{In fact, if the 
original phrase is shorter than the longest words in the dictionary,
we must include the ``one word'' solution. For example, if the target
phrase is \emph{gimp count}, there is a one word solution to be found: computing.}

\subsection{Recursion and bookkeeping}

At this point, we have the algorithm reasonably well in mind, if not
in hand. We take our collection of remainders and derive \Dpp from \Dp,
and reapply the testing of all the keys in \Dpp to $\mathfrak{R}$.
The dictionary rapidly becomes small. 

Additionally, we can exploit the fact that we are keeping track of
the \aw terms as we go to ensure that we do not test them more than
once, and this is where bookkeeping enters the picture. It seems
fairly natural to think of this as a forest of $n$-ary trees, where
the root node of each is a \aw term. For \aw terms that offer no
completion (\ie dead ends) we can saw these to the ground, and
experience has shown that dead ends will constitute the majority
of the \aw terms we try.

\newpage
\section{Programming}

This paper is being written in 2021, so the programming is 
done using Python~3. At this time, I am using Python~3.8, although
I do not think any of the features that first appear in Python~3.8
(such as the ``walrus operator,'' \lit{:=} ) are used in the code
that appears on GitHub.

This is a paper about anagrams, Python's data structures, and rolling
a few of our own data structures. It is not about PEP-8 style, sane
exception handling, type hints, nor how to organize code modules
in the project. With that warning, let's get started.

\subsection{Style of the dictionary}

The familiar \verb|/usr/share/dict/linux.words| file is based on the
Webster's Second International Dictionary. It has a number
of entries we do not need as its primary use is in spell-check. 
There are words with punctuation, and
acronyms that are all caps. Additionally, it has $1420$ words of three
letters, $25199$ words that start with a capital letter, and with 10230
words of five letters, it far exceeds Knuth's well established list
of $5757$ five letter words that is a corpus in the Stanford University
Graph Base.\footnote{\lit{https://www-cs-faculty.stanford.edu/$\sim$knuth/sgb.html}
Both the page and Knuth's book are well worth exploring.}

If you turn your attention to \lit{dictbuilder.py} you can get a
feel for the approach taken to support anagrams. I have chosen to
eliminate a large number of the dictionary entries by reading it
in a way that rids us of duplicates, capital letters, and punctuation
all at once, and I have chosen to supply an explicit list of $27$
two letter words rather than the $160$ in the dictionary. Feel
free to adjust the code to suit your use.

\lit{dictbuilder} creates two dictionaries.

\begin{enumerate}
\item a \lit{dict} (dictionary) whose keys are the words we have
in some way read from the dictionary file, and whose values are the
sorted strings of the letters in the word.

$$w \longrightarrow \overrightarrow{w}$$

Rather arbitrarily, this is termed the \emph{forward dictionary},
and in the code dictionaries of this type are usually referred
to by the sybolic name \verb|f_dict|.

\item a \lit{dict} whose keys are the sorted strings from above,
and whose values are a set$\slash$tuple of all the words from the
dictionary that can be made from this string of letters.

$$\overrightarrow{w} \longrightarrow (w_0, w_1, w_2, .. w_n)$$

This is termed the \emph{reverse dictionary}, and it is associated
with the symbolic name \verb|r_dict|. \inote{The container is represented
as a \lit{tuple}, but it is also a \lit{set} because each element
occurs exactly once.}

\end{enumerate}

From the standpoint of use in dictionaries, it is required that
both the keys and values be hashable, because the values become the
keys in the reversed dictionary. Consequently, we cannot use a
sorted \lit{list} of letters; it must be a \lit{str} or a \lit{tuple}.
Ordinary strings are the most convenient, particularly when 
printing the results.


\subsection{Going from words to sorted strings}

In the Python Standard Library there are wonders, and one of the
ones we will be using is \lit{collections}, and within it we will
start with the handy \lit{Counter} class. As the documentation
states, \lit{Counter} is a type of \lit{dict}.\footnote{The
documentation referred to here and throughout this paper is the
collection of web pages at \lit{docs.python.org}.  It is searchable,
well written, and accurate. You should not only use it, you should
prefer it.}

Referring to Figure~\ref{fig:anagramsymbols}, we can see that 

\begin{quote}
\small
\begin{verbatim}
>>> S = 'george flanagin'
>>> sorted_S = str(sorted([ _ for _ in S if _ != ' ' ]))
>>> counted_S = collections.Counter(S)
>>> counted_S
Counter({'g': 3, 'e': 2, 'a': 2, 'n': 2, 'o': 1, 'r': 1, 
    'f': 1, 'l': 1, 'i': 1})
\end{verbatim}
\end{quote}

\lit{Counter} defines a number of operations that are very close
to ideal for our use in abstracted algebra to deal with anagrams.
It is good that we get a head start, but we do need to put in
a bit of work to customize the \lit{Counter} for our purposes.  The
most direct route is the exploitation of Python's underlying object
model.

\lit{Counter} gives us a useful iterator named \lit{elements()} that
we can pass directly to the \lit{sorted} builtin.  

\begin{quote}
\small
\begin{verbatim}
>>> sorted(counted_S.elements())
['a', 'a', 'e', 'e', 'f', 'g', 'g', 'g', 'i', 'l', 'n', 'n', 'o', 'r']
\end{verbatim}
\end{quote}

As a subclass of \lit{dict}, \lit{Counter} has specialized the
\lit{update()} method using the plus (\lit{+}) operator, and it will do
exactly what we require to implement the method in our notation
written as $\oplus$. Unfortunately for our work, the \lit{subtract()}
function allows negative quantities,\footnote{The subclassed
\lit{update()} also allows for negative quantities, but when we are
adding objects whose count is greater than zero there is no risk
of getting a negative result.} which means we will need to modify
it slightly in our subclass for it to be an implementation of
$\ominus$.

The code in Figure~\ref{fig:countermods} (with most comments removed for bevity)
accomplishes the required changes in a class named \lit{CountedWord}:

\begin{enumerate}
\item As a subclass of \lit{Counter}, we get to use the builtin
methods.

\item In keeping with the spirit of the \lit{Counter} implementation,
we have superseded the meanings in the original class, so that \lit{a-b} does
not modify \lit{a}, and \lit{a -= b} is provided for the cases where that is
desired.

\item We have provided a \verb|__str__()| operator that returns the
sorted string of the characters in the counter. This is generally the
most useful case when we want to use a \lit{CountedWord} as 
a key. 

\end{enumerate}

\begin{figure}
\begin{framed}
\begin{quote}
\tiny
\begin{verbatim}
@total_ordering
class CountedWord(Counter):
    """
    Each word/phrase corresponds to one CountedWord representation of it.
    For example, CountedWord('georgeflanagin') is 'aaeefgggilnnor'. However,
    the same CountedWord may be a representation of many different words.

    The operators allow us to write code that is somewhat algebraic.
    """
    def __init__(self, s:str):
        """
        Add one class member, the as_str, which is a the word
        represented as a
        """
        Counter.__init__(self, s)
        self.as_str = "".join(sorted(self.elements()))


    def __eq__(self, other:Union[CountedWord,str]) -> bool:
        """
        if CountedWord(w1) == CountedWord(w2), then w1 and w2 are
        anagrams of each other.  For example CountedWord('loaf') ==
        CountedWord('foal').
        """
        if isinstance(other, str): other = CountedWord(other)
        return self.as_str == other.as_str


    def __le__(self, other:Union[CountedWord,str]) -> bool:
        """
        if shred1 <= shred2, then shred1 is in shred2
        """
        if isinstance(other, str): other=Counter(other)

        # Note that there are no zero-counts in the Counter's
        # dict. So all the v-s from self will be > 0.
        return all(other.get(c, 0) >= v for c, v in self.items())


    def __sub__(self, other:CountedWord) -> CountedWord:
        if isinstance(other, (str, Counter)): other = CountedWord(other)
        if other <= self:
            x = copy.copy(self)
            x.subtract(other)
            x.__clean()
            x.as_str = "".join(sorted(x.elements()))
            return x
        else:
            raise ValueError('RHS is not <= LHS')


    def __add__(self, other:CountedWord) -> CountedWord:
        if isinstance(other, (str, Counter)): other = CountedWord(other)
        x = copy.copy(self)
        x.update(other)
        x.as_str = "".join(sorted(self.elements()))
        return x


    def __clean(self) -> None:
        zeros = [ k for k in self if self[k] == 0 ]
        for k in zeros:
            self.pop(k)


    def __str__(self) -> str:
        """
        The contents, sorted, and as a string.
        """
        return self.as_str
\end{verbatim}
\end{quote}
\end{framed}
\normalsize
\caption{Example code for derived \lit{Counter} class}
\label{fig:countermods}
\end{figure}

\subsection{N-ary trees in Python}

GitHub is filled with trees for Python, and the reason is very
likely that Python (as of 3.8) has no native tree in its standard
library.  For anagrams, we need nothing complex like self-balancing
red-black trees, nor even B-trees. We need a flexible N-ary tree.

It is fairly easy to construct one starting from the idea behind
the \lit{defaultdict} from the \lit{collections} module, although
it does not quite work for us as delivered. We need to support all
the following abstractions:

\begin{enumerate}
\item From graph theory fundamentals, we know that a tree is a data
structure in which there is exactly one path from any node to another
node. This implies that a linear sequence is also a tree, although
not a very ornamental example.

\item An important property of a tree can be derived from the
definition: if we cut a tree into two pieces by removing the unique
connection between any two nodes, the resulting two data structures
are each trees. To look at it from the opposite direction (the
direction more useful for a recursive search like finding anagrams),
we can take any two trees and perform a grafting operation at a
node to join them into a single tree.

\end{enumerate}

The needed operations in Python can be accomplished with the
\lit{dict} because \lit{dict}s may contain \lit{dict}s as members.
From a programming standpoint, this operation needs to be as intuitive
as possible if we are to avoid common programming mistakes.

The single data structure for trees in this program is known as 
the \lit{SloppyTree}, a name that speaks for itself. It is directly
derived from \lit{dict}, with the following behaviors modified:

\begin{itemize}
\item \lit{SloppyTree} provides a \verb|__missing__| function that
creates a new key with the default value of an empty \lit{SloppyTree} in
the case where the key is not found. The native \lit{dict} function
raises a \lit{KeyError} when elements are not found; instead we want to
automatically add them. This behavior is also found in \lit{defaultdict}.

\item Consistent with \lit{class}-like behavior, \lit{SloppyTree} provides
the trio of member access operations: \verb|__getattr__|, \verb|__setattr__|,
and \verb|__delattr__|. These operations are not required for the 
case of anagrams, but \lit{SloppyTree} has other uses.

\item Consistent with our desire to write as little code as possible, 
\lit{SloppyTree} has a \verb|__str__| operator that invokes the \lit{pprint.pformat}
function to write out the contents in a way suitable for review.

\end{itemize}

As much as we benefit from the \lit{CountedWord} class, it is unfortunate
that they are not hashable, and therefore cannot be keys for the \lit{SloppyTree}
nor any other \lit{dict}-derived type. This is why the \verb|CountedWord.__str__| 
operator returns the extra class member \verb|as_str| --- we really do not want
to be constantly rebuilding them. 

\section{Using the Anagrammar}
\subsection{Command line parameters}

Launching the program produces a familiar style of command line usage 
message:

\small
\begin{verbatim}
usage: anagrammar [--help] [--cpu CPU] --dictionary DICTIONARY [--min-len MIN_LEN]
                  [--no-dups] [--none-of NONE_OF] [--order {0,1,2}] [-v {0,1,2,3}]
                  phrase [phrase ...]
anagrammar: error: the following arguments are required: phrase
\end{verbatim}
\normalsize

\begin{figure}
\small
\begin{verbatim}
positional arguments:
  phrase                The phrase. If it contains spaces, it must be in quotes.

optional arguments:
  -h, --help            show this help message and exit
  --cpu CPU             Set a maximum number of CPU seconds for execution.
  --dictionary DICTIONARY
                        Name of the dictionary of words, or a pickle of the
                        dictionary.
  --min-len MIN_LEN     Minimum length of any word in the anagram
  --no-dups             Disallow words that were in the original phrase.
  --none-of NONE_OF     Exclude all words in the given filename.
  --order {0,1,2}       Key ordering: 0: random, 1:shortest first, 2:longest first
  -v {0,1,2,3}, --verbose {0,1,2,3}
                        Be chatty about what is taking place -- on a scale of 0 to 3
\end{verbatim}
\normalsize
\caption{Anagrammar help text}
\label{fig:helptext}
\end{figure}

A polite inquiry with \lit{--help} provides slightly more information as shown
in Figure~\ref{fig:helptext}. Unfortunately, it is not possible to provide
all the information needed in just a line or two. Let's examine the options
in alphabetic order.

\begin{description}
\item[\lit{--cpu}] This option is primarily useful to prevent runaway searches
when the phrase being anagrammed is capable of making a large number of 
anagrams. Given that the algorithm is tail-recursive and single-threaded,
the processor time correlates well with the number of branches in the
tree.

\item[\lit{--dictionary}] Half the fun in this program is changing out
the dictionary of allowed words, and the other half is changing the
original phrase. Only the non-suffix part of the filename needs to
be given, and the Anagrammar will accept the name you give and assume
that the pair of dictionary files is named \lit{.forward} and \lit{.reversed}
 
\item[\lit{--min-len}] This parameter is the most significant one in
changing the results and altering the execution time. The default
value is \lit{2}, which is too low for the most interesting results. 

Note what happens when you allow a value of \lit{1} with the input
phrase ``george flanagin.'' Using the two one-letter words of English,
this amounts to running the algorithm twice with \lit{a} and \lit{i} 
extracted in turn on the remaining 13 letters. 

Note that there are problems at the other end, as well. For example,
suppose the input phrase has 18 letters, and the words must be six letters
long. With these initial conditions, the only possible combinations are three six-letter words
or two words where the pairs are \lit{(9, 9)}, \lit{(8, 10)}, \lit{(7, 11)}, and \lit{(6, 12)} letters.

\item[\lit{--nice}] Niceness is available for user programs, and the
higher the value, the more willing the OS will be to execute another
program. On the other hand, it may give the program a longer quantum and
fewer interruptions when it is running. 

\item[\lit{--no-dups}] When this parameter is present and the input
phrase is written as more than one word, \ie \emph{george flanagin} 
rather than \emph{georgeflanagin}, none of the words are allowed 
in the resulting solution set of anagrams.

\item[\lit{--none-of}] The value of this parameter is a string that
is interpreted as a filename containing white-space delimited words to
be removed from consideration. The methods for removing entries from
a pair of \lit{dict}s with thousands of entries is a kludge, and this 
option does not exist as an option to creating a custom dictionary.
See Section~\ref{sec:dictbuilder} for information about constructing
custom dictionaries.

\item[\lit{--order}] This parameter exists primarily to do demonstrations
of the importance of the search order for anagrams. The options are:

\begin{description}
\item[0] random order, created by trying the possible keys in 
an order provided by the Python builtin, \lit{random.sample}. The
results are not repeatable, nor can the sample method be tuned.

\item[1] shortest keys first. This is the ``correct''
operation.

\item[2] longest keys first. This method produces only a subset of
the possible anagrams, but it is quite useful to illustrate the
importance of search order.

\end{description}

\item[\lit{--verbose}] At this time, only \lit{--verbose 3} has a profound
effect on output. The high setting of verbosity creates an effective
flow trace showing the keys being tried and exhausted, and the
recursions noted.

\end{description}

\subsection{Example outputs}

\subsubsection{Statistics}

As mentioned in Section~\ref{sec:bigpicture}, the number of anagrams
can become large, quickly, and for a variety of reasons -- the
dictionary used, the letters making up the phrase. Therefore, the
example that is used in this paper is a little contrived.

First, let's take a look at the standard output that is created
while the program is running in its standard mode as shown
in Figure~\ref{fig:stdoutput}.  Note that the
results will vary widely across operating systems and processors.

\begin{figure}
\small
\begin{verbatim}
Initial pruning: 880 keys representing 1054 words.

D | branch |  dead  |  user  |  sys   |  page  |  I/O  | WAIT | USEDQ |  Tails  |
  | evals  |  ends  |  secs  |  secs  | faults |  sig  |  sig |       |         |
---+--------+--------+--------+--------+--------+-------+------+-------+---------|
  1    26017     8186     2.18     0.07    32829       0      7   1269      8640

26017 branches in the tree. 8186 dead ends. Max depth 3.
\end{verbatim}
\normalsize
\caption{Standard program output.}
\label{fig:stdoutput}
\end{figure}

On the line that precedes the table, we can read the size of the vocabulary
that has qualified for use in the search. In this example, there are 880 keys
(sorted strings) that represent 1054 words taken from the dictionary. 
The output is updated every 100 branch evaluations as the program runs.

The left-most column shows the current recursion depth, and it is more for
evidence of operation than any practical use. It will always be \lit{1} when
the program terminates. The other parameters are more significant.

\begin{description}
\item[branch evals] The count of the number of attempts to add a branch
to the tree.

\item[dead ends] The number of failed branches during tree construction.

\item[user secs] The number of seconds the program has been executing.
This value added to the \lit{sys secs} value gives the ``clock time''
of operation.

\item[sys secs] The number of seconds the system has been performing
tasks on behalf of the program.

\item[page faults] The number of times the program has had to request 
a page from memory. Typically, this number gains most of its value when the
dictionaries are read into memory, and it grows only slightly as the
tree of results grows.

\item[I/O sig] The number of times the program has stopped to perform
I/O with storage (disc), as opposed to memory.

\item[WAIT sig] The number of times the program has been told to wait
by the operating system.

\item[USEDQ] All operating systems allow a program to run for a specific
amount of time, known as the \emph{quantum}, at which point the scheduler
interrupts to be sure there is nothing more important for the current 
core to do. The number of interruptions is highly dependent on the 
CPU and the OS, but the ratio of \lit{USEDQ / (WAIT + I/O)} is a 
clear indication of the degree to which the task is ``CPU-bound.''

\item[Tails] The final remainder is called a \emph{tail}, a suitable
name because the algorithm is \emph{tail recursive}. At the end
of each failure to find an anagram, the terminating event is a 
tail that makes no word in the dictionary, or a tail that has
been discovered to not be decomposable into two or more words
in the dictionary. To avoid traversing the same tails more than
once, each new tail is added to a Python \lit{set}. The number
shown in this column is the final total.

\end{description}

\subsubsection{Anagrams of a sample phrase}

Using \emph{sample phrase} as the sample phrase, and setting the minimum
length to 4, we get interesting results when using the full Linux 
spelling dictionary. The \anagrammar's normal output is shown in Figure~\ref{fig:samplephrase}.
The less interesting, but more easily understood collection of anagrams from the MIT 10000 most
common words is shown in Figure~\ref{fig:smallsample}.


\begin{figure}
\footnotesize
\begin{verbatim}
anagram --min-len 4 --dict mit10000 "sample phrase"

['sample', 'phrase']
Initial pruning: 86 keys representing 98 words.

 D | branch |  dead  |  user  |  sys   |  page  |  I/O  | WAIT | USEDQ |  Tails  |
   | evals  |  ends  |  secs  |  secs  | faults |  sig  |  sig |       |         |
---+--------+--------+--------+--------+--------+-------+------+-------+---------|
  1      119       95     0.11     0.03     4935       0      6    103       102

119 branches in the tree. 95 dead ends. Max depth 3.
{   'maple': 'phrases',
    'peas': {('arms', 'mars'): 'help'},
    'phrase': 'sample',
    'sphere': 'plasma',
    ('males', 'salem', 'meals'): 'perhaps',
    ('phases', 'shapes'): 'palmer'}

\end{verbatim}
\normalsize
\caption{Anagrams of \emph{sample phrase} using MIT 10000 word dictionary}
\label{fig:smallsample}
\end{figure}

The tree is printed so that the anagrams may be located by looking at the 
lines of the report. Looking at Figure~\ref{fig:smallsample}, we see that 
\emph{maple} and \emph{phrases} is an easy anagram, the kind that Lisa Simpson
posited for ``Jeremy Irons'' \ldots ``Jeremy's Iron,'' missing ``minor jersey.''

The next line features a three word anagram, \emph{peas}, \emph{help}, and 
either of the self-anagrams, \emph{arms} and \emph{mars}. Since we did not
exclude the original terms, they appear in the third line. Contrast this 
with the penultimate anagram in the list constructed from the larger dictionary.
It shows that \emph{phrase} has two self-anagrams in the dictionary, 
\emph{shaper} and \emph{seraph}.


\begin{figure}
\footnotesize
\begin{verbatim}
anagram --min-len 4 --dict words "sample phrase"

['sample', 'phrase']
Initial pruning: 262 keys representing 448 words.

 D | branch |  dead  |  user  |  sys   |  page  |  I/O  | WAIT | USEDQ |  Tails  |
   | evals  |  ends  |  secs  |  secs  | faults |  sig  |  sig |       |         |
---+--------+--------+--------+--------+--------+-------+------+-------+---------|
  1      344      267     1.21     0.07    32558       0      3    363       301

344 branches in the tree. 267 dead ends. Max depth 3.
{   'alpha': 'empress',
    'apple': 'smasher',
    'harem': ('sapples', 'papless'),
    'helms': 'sappare',
    'papess': 'harmel',
    'peeps': 'marshal',
    'phases': ('palmer', 'lamper', 'relamp'),
    'remap': 'hapless',
    'slash': 'empaper',
    'spasm': 'preheal',
    'sperma': 'alephs',
    'splash': 'ampere',
    ('hames', 'shame'): 'slapper',
    ('haps', 'hasp', 'pash'): 'resample',
    ('hassel', 'hassle'): ('pampre', 'mapper', 'pamper'),
    ('heaps', 'phase', 'shape'): ('lampers', 'sampler'),
    ('lames', 'males', 'meals'): ('prehaps', 'perhaps'),
    ('lamps', 'plasm', 'palms', 'psalm'): ('reshape', 'rephase'),
    ('lams', 'slam', 'alms'): 'preshape',
    ('peas', 'spae', 'apse', 'apes', 'pase'): {   'harp': ('mels', 'elms'),
                                                  'hemp': 'lars',
                                                  'lash': 'perm',
                                                  'marl': 'pehs',
                                                  'reps': 'halm',
                                                  ('haps', 'hasp', 'pash'): 'merl',
                                                  ('mars', 'arms', 'rams'): 'help',
                                                  ('palm', 'lamp'): ('hers', 'resh'),
                                                  ('raps', 'pars', 'spar', 'rasp'): 'helm',
                                                  ('slap', 'pals', 'alps', 'salp', 'laps'): 
                                                        'herm',
                                                  ('spam', 'pams', 'samp', 'maps', 'amps'): 
                                                        ('herl', 'lehr')},
    ('plasma', 'lampas'): ('sphere', 'herpes'),
    ('seraph', 'phrase', 'shaper'): 'sample',
    ('sperm', 'perms'): 'phaseal'}
\end{verbatim}
\normalsize
\caption{Anagrams of \emph{sample phrase}}
\label{fig:samplephrase}
\end{figure}

\section{Building dictionaries}

The source code contains a single file named \lit{dictbuilder.py}. It is both
a standalone program and an \lit{import} for this project because it contains
the function \lit{dictloader}. The dictionaries it builds are stored as 
Python pickles, and it offers useful command line switches to control
the way the dictionaries are constructed.

\subsection{Basic operations}

\begin{figure}
\small
\begin{verbatim}
Usage: dictbuilder [-h] [-d] -i INPUT [INPUT ...] [-n PROPERNOUNS] outfile

A program to maintain dictionaries used in anagrammar.

positional arguments:
  outfile               Name of the dictionary to be written (minus the suffixes).

optional arguments:
  -h, --help            show this help message and exit
  -b, --bare            Use only the words in the input dictionary rather than the 
                        built-in 2, 3, 4, and 5 letter words.
  -i INPUT [INPUT ...], --input INPUT [INPUT ...]
                        The name[s] of the input dictionaries.
  -n PROPERNOUNS, --propernouns PROPERNOUNS
                        If used, exclude the words found in the file (presumed to 
                        be proper nouns)
\end{verbatim}
\normalsize
\caption{Command line options for the dictbuilder}
\label{fig:dictbuilderoptions}
\end{figure}

Running the \lit{dictbuilder} is much like running the \anagrammar. The help
text is shown in Figure~\ref{fig:dictbuilderoptions}. The options offer
considerable flexibility for making custom dictionaries. \inote{When a 
dictionary is loaded by the \anagrammar, the entire file is read. There is 
no option to inspect the dictionary while it is being read.} 

\begin{description}
\item[\lit{--bare}] Normal operation is to use the lists of short words that
are a part of the project. The 1, 2, and 3-letter words are in the source
code itself, represented as \lit{frozenset}s of \lit{str}. The 4-letter
words are in the file \lit{four.letter.words} and the 5-letter words 
are those of Donald Knuth, stored in the file \lit{knuths.5757.five.letter.words.txt}.
These may not be the ones you want to use, primarily because Knuth's list of
words is unusually large, exceeding even the Webster's International dictionary.

\item[\lit{--input}] This argument can be repeated, so that you may
build a dictionary from several, merged sources. Normal operation is effectively
the following:

\begin{quote}
\small
\lit{-i four.letter.words -i knuths.5757.five.letter.words -i yourownfile}
\end{quote}

\item[\lit{--propernouns}] If this option is not present, all the input
dictionaries are used \emph{as is}. If it is present, it removes all the 
words that appear in the dictionaries with an initial capital letter, and
then removes all the words in the filename associated with the option, 
regardless of whether they are capitalized.

\item[outfile] The name should be supplied without suffixes, because the
construction process creates the two dictionaries as separate files,
providing its own suffixes, \lit{.forward} and \lit{.reversed}.

\end{description}

\subsection{Methods}

The building of the dictionary takes place with the steps in the
logical order to avoid conflicts. The so-called \emph{forward} dictionary
is built first, with the words read from the various sources as the
keys, and the letters that make up each word, sorted in the standard
lexicographic way, as the values.

The construction of the reversed dictionary requires more finesse.
The construction iterates over the values of the forward dictionary to
use as keys, and the value is a \lit{defaultdict} of \lit{list}, to which the 
corresponding words are appended. Once constructed, neither dictionary
is mutable, so the values in the reversed dictionary are transformed into
\lit{tuple}s before the pickling process.

Pickles are generally faster to read than write, and they have the
advantage that type information is stored with the data, allowing
in memory data structures to be populated directly by reading the
pickle. 



\vfill
\begin{figure}[b]
    \footnotesize
    \setlength\fboxsep{1cm}
    \setlength\fboxrule{0pt}
        {\raisebox{-1.2cm}{\includegraphics[width=0.14\textwidth]{urlogo.pdf}}}
    \hfill{}
    \begin{tabular}{ll}
        \multicolumn{2}{l}{\textbf{George Flanagin, Provost Office}}\\
        \multicolumn{2}{l}{\textbf{Data Analysis \& Data Science}}\\
        \hline \\ [-1.9ex]
        \textbf{Phone:}&+1.804.287.6392\\
        \textbf{Address:}&Richmond Hall, Office 104\\
        &114 UR Drive\\
        &\UR\\
        &Richmond, VA 23173\\
        \textbf{Email:}&gflanagin@richmond.edu\\ 
        \textbf{ORCID}&0000-0002-2084-5831\\
    \end{tabular}
\end{figure}

\end{document}


























